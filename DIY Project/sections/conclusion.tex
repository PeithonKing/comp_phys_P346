\section{Conclusion}
	\begin{itemize}
		\item \textbf{Compression Quality:} The method of image compression using SVD is quite backdated in the modern world. Much better methods like \texttt{JPEG}, \texttt{JPEG2000}, \texttt{JPEG XR}, etc. have been developed. Using SVD, we saw that to keep most of the information in the image, we need to have a very high compression ratio. This is not the case with the modern methods. They can compress the image to a much lower ratio and still retain most of the information. For example in \texttt{JPEG}, our monkey image is 348KB $(1360\times1360\;pixels)$. This gives a compression factor of $\frac{348\times1024}{1360\times1360}=19\%$. This is a very good compression factor. But the image has all the imformation. This is not the case with SVD. We saw that to keep most of the information, we need a compression factor of $50\%$ or more. This is not practical in the modern world. However, the method of SVD is still used in some applications like image denoising, image inpainting, etc. where the quality of the image is not a major concern.
		
		\item \textbf{Advantages over better compression algorithms:} In this method, we are storing parts of the U, S and V matrices, which can be used to reconstruct the original image. However, for image processing purpose, we have an advantage of using SVD over other methods. The U, S and V matrices can be directly fed into the image processing algorithms for \textbf{feature extraction} (feature extraction is the first step for most image processing algorithms). This is not possible in other methods like \texttt{JPEG}, \texttt{JPEG2000}, \texttt{JPEG XR}, etc. where the image is compressed to a single file. This is because, in those algorithms, the compression is such that, feature extraction becomes impossible without reconstructing the original image. \textbf{This saves a lot of time and memory while doing image processing with those images.}
		
		\item \textbf{Combining other methods:} We saw that the SVD method is not very practical in real world application. On the other hand, if we combine it will other processes, we can have a very good compression ratio.
	\end{itemize}