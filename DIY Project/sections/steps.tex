\section{Steps to Calculate SVD}

	Calculating SVD of a matrix involves finding the eigenvalues and eigenvectors of the matrix. The \textbf{pseudocode} is as follows:

\vspace{5mm}

\noindent function SVD(A)\{
	\vspace{-3mm}
	\begin{enumerate}
		% \item if A is not a square matrix, add sufficient zeros as rows or columns to make it a square matrix
		\item Find eigenvalues of $A^TA$. \textbf{Note:} $A^TA$ is a square matrix.
		\item $S = (\sqrt{\lambda_i}\;|\;lambda_i \in eigenvalues$)
		\item Add zeros to $S$ to make it a square, diagonal matrix.
		\item V = eigenvectors of $A^TA$
		\item U = eigenvectors of $AA^T$
		\item return $U$, $S$, $V^T$
	\end{enumerate}

	\vspace{-3mm}
\noindent \}
